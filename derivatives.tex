% Converted from derivatives.md
\documentclass[12pt]{article}
\usepackage[utf8]{inputenc}
\usepackage{amsmath,amssymb}
\usepackage[a4paper,margin=1in]{geometry}
\usepackage{hyperref}
\usepackage{booktabs}
\usepackage{array}
	itle{Derivatives: A Comprehensive Guide}
\author{}
\date{}

\begin{document}
\maketitle
	ableofcontents
\clearpage

\section{Introduction}

The derivative is one of the most important concepts in calculus. It measures how a function changes as its input changes. In other words, it represents the rate of change or the slope of a function at a particular point.

The derivative has numerous applications in physics, engineering, economics, and many other fields.

\section{Definition of Derivative}

\subsection{Formal Definition}

The derivative of a function $f(x)$ at a point $x = a$ is defined as:
\begin{equation}
f'(a) = \lim_{h \to 0} \frac{f(a+h) - f(a)}{h}
\end{equation}

Alternatively, using the limit definition:
\begin{equation}
f'(x) = \lim_{\Delta x \to 0} \frac{\Delta y}{\Delta x} = \lim_{\Delta x \to 0} \frac{f(x + \Delta x) - f(x)}{\Delta x}
\end{equation}

\subsection{Notation}

The derivative of $f(x)$ can be denoted as:
\begin{itemize}
	\item $f'(x)$ (Lagrange notation)
	\item $\dfrac{df}{dx}$ (Leibniz notation)
	\item $\dfrac{d}{dx}f(x)$ (operator notation)
	\item $D_x f(x)$ (operator notation)
\end{itemize}

\subsection{Geometric Interpretation}

The derivative represents the slope of the tangent line to the curve $y = f(x)$ at a given point. A positive derivative means the function is increasing, while a negative derivative means it's decreasing.

\section{Rules of Differentiation}

\subsection{Power Rule}

For $f(x) = x^n$ where $n$ is any real number:

\begin{equation}
\frac{d}{dx}(x^n) = nx^{n-1}
\end{equation}

	extbf{Example:} $\dfrac{d}{dx}(x^3) = 3x^2$

\subsection{Sum and Difference Rule}

For functions $u(x)$ and $v(x)$:
\begin{align}
\frac{d}{dx}[u(x) + v(x)] &= \frac{du}{dx} + \frac{dv}{dx} \\
\frac{d}{dx}[u(x) - v(x)] &= \frac{du}{dx} - \frac{dv}{dx}
\end{align}

\subsection{Product Rule}

For functions $u(x)$ and $v(x)$:
\begin{equation}
\frac{d}{dx}[u(x) \cdot v(x)] = u'(x) \cdot v(x) + u(x) \cdot v'(x)
\end{equation}

Or in short form: $(uv)' = u'v + uv'$

\subsection{Quotient Rule}

For functions $u(x)$ and $v(x)$ where $v(x) \neq 0$:
\begin{equation}
\frac{d}{dx}\left[\frac{u(x)}{v(x)}\right] = \frac{u'(x) \cdot v(x) - u(x) \cdot v'(x)}{[v(x)]^2}
\end{equation}

Or in short form: $\left(\dfrac{u}{v}\right)' = \dfrac{u'v - uv'}{v^2}$

\subsection{Chain Rule}

For a composite function $y = f(g(x))$:
\begin{equation}
\frac{dy}{dx} = \frac{dy}{du} \cdot \frac{du}{dx}
\end{equation}

Where $u = g(x)$.

\section{Common Derivatives}

\begin{table}[h!]
\centering
\begin{tabular}{@{} l c @{} }
	oprule
Function & Derivative \\
\midrule
$c$ (constant) & $0$ \\
$x^n$ & $nx^{n-1}$ \\
$e^x$ & $e^x$ \\
$a^x$ & $a^x \ln(a)$ \\
$\ln(x)$ & $\dfrac{1}{x}$ \\
$\log_a(x)$ & $\dfrac{1}{x \ln(a)}$ \\
$\sin(x)$ & $\cos(x)$ \\
$\cos(x)$ & $-\sin(x)$ \\
$\tan(x)$ & $\sec^2(x)$ \\
$\cot(x)$ & $-\csc^2(x)$ \\
$\sec(x)$ & $\sec(x)\tan(x)$ \\
$\csc(x)$ & $-\csc(x)\cot(x)$ \\
\bottomrule
\end{tabular}
\caption{Common derivative formulas}
\end{table}

\section{Applications of Derivatives}

\subsection{Rate of Change}

Derivatives measure how quantities change over time. For example, velocity is the derivative of position with respect to time.

\subsection{Optimization}

Finding maximum and minimum values of functions. This is crucial in business (maximizing profit), physics (finding equilibrium points), and engineering.

\subsection{Curve Sketching}

The first derivative tells us where the function is increasing or decreasing. The second derivative tells us about concavity.

\subsection{Related Rates}

Problems where multiple quantities are related and change with respect to time.

\subsection{Physics}

\begin{itemize}
	\item \textbf{Velocity:} $v = \dfrac{ds}{dt}$ (derivative of position)
	\item \textbf{Acceleration:} $a = \dfrac{dv}{dt}$ (derivative of velocity)
	\item \textbf{Force:} $F = \dfrac{dp}{dt}$ (derivative of momentum)
\end{itemize}

\section{Examples}

\subsection{Example 1: Power Rule}

Find the derivative of $f(x) = 5x^4 - 3x^2 + 7$.

	extbf{Solution:}
\begin{equation}
f'(x) = 5 \cdot 4x^3 - 3 \cdot 2x + 0 = 20x^3 - 6x
\end{equation}

\subsection{Example 2: Product Rule}

Find the derivative of $f(x) = (2x + 1)(x^2 - 3)$.

	extbf{Solution:}
Let $u = 2x + 1$ and $v = x^2 - 3$, so $u' = 2$ and $v' = 2x$.
Using the product rule:
\begin{align}
f'(x) &= 2(x^2 - 3) + (2x + 1)(2x) \\
&= 2x^2 - 6 + 4x^2 + 2x \\
&= 6x^2 + 2x - 6
\end{align}

\subsection{Example 3: Chain Rule}

Find the derivative of $f(x) = (3x^2 + 2)^5$.

	extbf{Solution:}
Let $u = 3x^2 + 2$, then $\dfrac{du}{dx} = 6x$.
Using the chain rule:
\begin{equation}
f'(x) = 5(3x^2 + 2)^4 \cdot 6x = 30x(3x^2 + 2)^4
\end{equation}

\subsection{Example 4: Quotient Rule}

Find the derivative of $f(x) = \dfrac{x^2 + 1}{x - 1}$.

	extbf{Solution:}
Let $u = x^2 + 1$ and $v = x - 1$, then $u' = 2x$ and $v' = 1$.
Using the quotient rule:
\begin{align}
f'(x) &= \frac{2x(x - 1) - (x^2 + 1)(1)}{(x - 1)^2} \\
&= \frac{2x^2 - 2x - x^2 - 1}{(x - 1)^2} \\
&= \frac{x^2 - 2x - 1}{(x - 1)^2}
\end{align}

\section{Conclusion}

Understanding derivatives is fundamental to calculus and has applications throughout science, engineering, and economics. By mastering the rules and concepts presented here, you'll be well-equipped to solve a wide variety of problems involving rates of change and optimization.

\end{document}

