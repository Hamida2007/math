\documentclass[12pt]{article}
\usepackage[utf8]{inputenc}
\usepackage{amsmath,amssymb}
\usepackage[a4paper,margin=1in]{geometry}
\usepackage{hyperref}
\usepackage{booktabs}
\usepackage{array}

\title{Derivatives: A Comprehensive Guide}
\author{}
\date{}

\begin{document}
\maketitle
\tableofcontents
\clearpage

\section{Introduction}

The derivative is one of the most important concepts in calculus. It measures how a function changes as its input changes. In other words, it represents the rate of change or the slope of a function at a particular point.

The derivative has numerous applications in physics, engineering, economics, and many other fields.

\section{Definition of Derivative}

\subsection{Formal Definition}

The derivative of a function $f(x)$ at a point $x = a$ is defined as:
\begin{equation}
f'(a) = \lim_{h \to 0} \frac{f(a+h) - f(a)}{h}
\end{equation}

Alternatively, using the limit definition:
\begin{equation}
f'(x) = \lim_{\Delta x \to 0} \frac{\Delta y}{\Delta x} = \lim_{\Delta x \to 0} \frac{f(x + \Delta x) - f(x)}{\Delta x}
\end{equation}

\subsection{Notation}

The derivative of $f(x)$ can be denoted as:
\begin{itemize}
  \item $f'(x)$ (Lagrange notation)
  \item $\dfrac{df}{dx}$ (Leibniz notation)
  \item $\dfrac{d}{dx}f(x)$ (operator notation)
  \item $D_x f(x)$ (operator notation)
\end{itemize}

\subsection{Geometric Interpretation}

The derivative represents the slope of the tangent line to the curve $y = f(x)$ at a given point. A positive derivative means the function is increasing, while a negative derivative means it's decreasing.

\section{Rules of Differentiation}

\subsection{Power Rule}

For $f(x) = x^n$ where $n$ is any real number:
\begin{equation}
\frac{d}{dx}(x^n) = nx^{n-1}
\end{equation}

\textbf{Example:} $\dfrac{d}{dx}(x^3) = 3x^2$

\subsection{Sum and Difference Rule}

For functions $u(x)$ and $v(x)$:
\begin{align}
\frac{d}{dx}[u(x) + v(x)] &= \frac{du}{dx} + \frac{dv}{dx} \\
\frac{d}{dx}[u(x) - v(x)] &= \frac{du}{dx} - \frac{dv}{dx}
\end{align}

\subsection{Product Rule}

For functions $u(x)$ and $v(x)$:
\begin{equation}
\frac{d}{dx}[u(x) \cdot v(x)] = u'(x) \cdot v(x) + u(x) \cdot v'(x)
\end{equation}

Or in short form: $(uv)' = u'v + uv'$

\subsection{Quotient Rule}

For functions $u(x)$ and $v(x)$ where $v(x) \neq 0$:
\begin{equation}
\frac{d}{dx}\left[\frac{u(x)}{v(x)}\right] = \frac{u'(x) \cdot v(x) - u(x) \cdot v'(x)}{[v(x)]^2}
\end{equation}

Or in short form: $\left(\dfrac{u}{v}\right)' = \dfrac{u'v - uv'}{v^2}$

\subsection{Chain Rule}

For a composite function $y = f(g(x))$:
\begin{equation}
\frac{dy}{dx} = \frac{dy}{du} \cdot \frac{du}{dx}
\end{equation}

Where $u = g(x)$.

\section{Common Derivatives}

This section provides a reference table of derivatives for common functions, along with brief explanations of each.

\subsection{Basic Derivatives}

\begin{table}[h!]
\centering
\begin{tabular}{@{} l c l @{} }
\toprule
Function & Derivative & Notes \\
\midrule
$c$ (constant) & $0$ & The derivative of any constant is zero \\
$x$ & $1$ & The derivative of $x$ is always 1 \\
$x^n$ & $nx^{n-1}$ & Power rule applies for any real $n$ \\
\bottomrule
\end{tabular}
\caption{Basic polynomial derivatives}
\end{table}

\subsection{Exponential and Logarithmic Derivatives}

\begin{table}[h!]
\centering
\begin{tabular}{@{} l c l @{} }
\toprule
Function & Derivative & Notes \\
\midrule
$e^x$ & $e^x$ & The exponential function is its own derivative \\
$a^x$ & $a^x \ln(a)$ & For any base $a > 0$, $a \neq 1$ \\
$\ln(x)$ & $\dfrac{1}{x}$ & Natural logarithm; domain: $x > 0$ \\
$\log_a(x)$ & $\dfrac{1}{x \ln(a)}$ & Logarithm with base $a$ \\
\bottomrule
\end{tabular}
\caption{Exponential and logarithmic derivatives}
\end{table}

\textbf{Examples:}
\begin{itemize}
  \item $\dfrac{d}{dx}(e^x) = e^x$
  \item $\dfrac{d}{dx}(2^x) = 2^x \ln(2) \approx 0.693 \cdot 2^x$
  \item $\dfrac{d}{dx}(\ln(x)) = \dfrac{1}{x}$ for $x > 0$
\end{itemize}

\subsection{Trigonometric Derivatives}

\begin{table}[h!]
\centering
\begin{tabular}{@{} l c @{} }
\toprule
Function & Derivative \\
\midrule
$\sin(x)$ & $\cos(x)$ \\
$\cos(x)$ & $-\sin(x)$ \\
$\tan(x)$ & $\sec^2(x)$ \\
$\cot(x)$ & $-\csc^2(x)$ \\
$\sec(x)$ & $\sec(x)\tan(x)$ \\
$\csc(x)$ & $-\csc(x)\cot(x)$ \\
\bottomrule
\end{tabular}
\caption{Trigonometric derivatives}
\end{table}

\textbf{Key observations:}
\begin{itemize}
  \item The derivative of $\sin(x)$ is $\cos(x)$ and vice versa (with a sign change)
  \item Derivatives of reciprocal trig functions include the original function in the result
  \item All trig derivatives involve other trig functions
\end{itemize}

\subsection{Inverse Trigonometric Derivatives}

\begin{table}[h!]
\centering
\begin{tabular}{@{} l c l @{} }
\toprule
Function & Derivative & Domain \\
\midrule
$\arcsin(x)$ & $\dfrac{1}{\sqrt{1-x^2}}$ & $|x| < 1$ \\
$\arccos(x)$ & $-\dfrac{1}{\sqrt{1-x^2}}$ & $|x| < 1$ \\
$\arctan(x)$ & $\dfrac{1}{1+x^2}$ & All real $x$ \\
\bottomrule
\end{tabular}
\caption{Inverse trigonometric derivatives}
\end{table}

\subsection{Hyperbolic Derivatives}

\begin{table}[h!]
\centering
\begin{tabular}{@{} l c @{} }
\toprule
Function & Derivative \\
\midrule
$\sinh(x)$ & $\cosh(x)$ \\
$\cosh(x)$ & $\sinh(x)$ \\
$\tanh(x)$ & $\operatorname{sech}^2(x)$ \\
\bottomrule
\end{tabular}
\caption{Hyperbolic derivatives}
\end{table}

\subsection{Absolute Value and Piecewise Derivatives}

\begin{table}[h!]
\centering
\begin{tabular}{@{} l c l @{} }
\toprule
Function & Derivative & Notes \\
\midrule
$|x|$ & $\operatorname{sgn}(x)$ & Undefined at $x = 0$ \\
$\sqrt{x}$ & $\dfrac{1}{2\sqrt{x}}$ & Domain: $x > 0$ \\
$\sqrt[n]{x}$ & $\dfrac{1}{n\sqrt[n]{x^{n-1}}}$ & For positive integer $n$ \\
\bottomrule
\end{tabular}
\caption{Absolute value and root derivatives}
\end{table}

\textbf{Explanation:}
\begin{itemize}
  \item The derivative of $|x|$ is the sign function $\operatorname{sgn}(x) = \begin{cases} -1 & x < 0 \\ 0 & x = 0 \\ 1 & x > 0 \end{cases}$
  \item For square root: $\dfrac{d}{dx}(\sqrt{x}) = \dfrac{1}{2\sqrt{x}} = \dfrac{1}{2}x^{-1/2}$ (using power rule with $n = 1/2$)
  \item For $n$-th roots, use the power rule: $\dfrac{d}{dx}(x^{1/n}) = \dfrac{1}{n}x^{(1/n)-1}$
\end{itemize}

\subsection{Composition Rules for Derivatives}

When derivatives are combined, we use the following composition rules:

\begin{table}[h!]
\centering
\begin{tabular}{@{} l c l @{} }
\toprule
Rule & Formula & Example \\
\midrule
Sum Rule & $(u + v)' = u' + v'$ & $\dfrac{d}{dx}(x^2 + \sin x) = 2x + \cos x$ \\
\midrule
Difference Rule & $(u - v)' = u' - v'$ & $\dfrac{d}{dx}(e^x - x^3) = e^x - 3x^2$ \\
\midrule
Constant Multiple & $(cu)' = cu'$ & $\dfrac{d}{dx}(5\cos x) = -5\sin x$ \\
\midrule
Product Rule & $(uv)' = u'v + uv'$ & $\dfrac{d}{dx}(x \cdot e^x) = e^x + xe^x$ \\
\midrule
Quotient Rule & $\left(\dfrac{u}{v}\right)' = \dfrac{u'v - uv'}{v^2}$ & $\dfrac{d}{dx}\left(\dfrac{\sin x}{x}\right) = \dfrac{x\cos x - \sin x}{x^2}$ \\
\midrule
Chain Rule & $(f \circ g)' = f'(g) \cdot g'$ & $\dfrac{d}{dx}(\sin(x^2)) = 2x\cos(x^2)$ \\
\bottomrule
\end{tabular}
\caption{Composition and combination rules for derivatives}
\end{table}

\subsection{Practical Tips for Computing Derivatives}

\begin{enumerate}
  \item \textbf{Identify the structure:} Determine if the function is a sum, product, quotient, or composition.
  \item \textbf{Apply the appropriate rule:} Use sum, product, quotient, or chain rule as needed.
  \item \textbf{Use tables:} Refer to the common derivatives table for basic functions.
  \item \textbf{Simplify:} Always simplify the final answer by combining like terms and factoring when possible.
  \item \textbf{Check domain restrictions:} Remember where derivatives may be undefined (e.g., $\ln(x)$ only for $x > 0$).
\end{enumerate}

\textbf{Common mistakes to avoid:}
\begin{itemize}
  \item Forgetting to apply the chain rule to composite functions
  \item Misapplying the product rule (it's $u'v + uv'$, not $(u'v)'$ or $(uv')'$)
  \item Ignoring domain restrictions of the original function
  \item Forgetting constant factors in the numerator of quotient rule results
\end{itemize}

\begin{table}[h!]
\centering
\begin{tabular}{@{} l c @{} }
\toprule
Function & Derivative \\
\midrule
$c$ (constant) & $0$ \\
$x^n$ & $nx^{n-1}$ \\
$e^x$ & $e^x$ \\
$a^x$ & $a^x \ln(a)$ \\
$\ln(x)$ & $\dfrac{1}{x}$ \\
$\log_a(x)$ & $\dfrac{1}{x \ln(a)}$ \\
$\sin(x)$ & $\cos(x)$ \\
$\cos(x)$ & $-\sin(x)$ \\
$\tan(x)$ & $\sec^2(x)$ \\
$\cot(x)$ & $-\csc^2(x)$ \\
$\sec(x)$ & $\sec(x)\tan(x)$ \\
$\csc(x)$ & $-\csc(x)\cot(x)$ \\
$\arcsin(x)$ & $\dfrac{1}{\sqrt{1-x^2}}$ \\
$\arccos(x)$ & $-\dfrac{1}{\sqrt{1-x^2}}$ \\
$\arctan(x)$ & $\dfrac{1}{1+x^2}$ \\
$\sinh(x)$ & $\cosh(x)$ \\
$\cosh(x)$ & $\sinh(x)$ \\
$\tanh(x)$ & $\operatorname{sech}^2(x)$ \\
\bottomrule
\end{tabular}
\caption{Comprehensive list of common derivatives}
\end{table}

\section{Applications of Derivatives}

\subsection{Rate of Change}

Derivatives measure how quantities change over time. For example, velocity is the derivative of position with respect to time.

\subsection{Optimization}

Finding maximum and minimum values of functions. This is crucial in business (maximizing profit), physics (finding equilibrium points), and engineering.

\subsection{Curve Sketching}

The first derivative tells us where the function is increasing or decreasing. The second derivative tells us about concavity.

\subsection{Related Rates}

Problems where multiple quantities are related and change with respect to time.

\subsection{Physics}

\begin{itemize}
  \item \textbf{Velocity:} $v = \dfrac{ds}{dt}$ (derivative of position)
  \item \textbf{Acceleration:} $a = \dfrac{dv}{dt}$ (derivative of velocity)
  \item \textbf{Force:} $F = \dfrac{dp}{dt}$ (derivative of momentum)
\end{itemize}

\section{Examples}

\subsection{Example 1: Power Rule}

Find the derivative of $f(x) = 5x^4 - 3x^2 + 7$.

\textbf{Solution:}

We apply the power rule to each term separately:
\begin{itemize}
  \item For $5x^4$: The exponent is 4, so $\dfrac{d}{dx}(5x^4) = 5 \cdot 4x^{4-1} = 20x^3$
  \item For $-3x^2$: The exponent is 2, so $\dfrac{d}{dx}(-3x^2) = -3 \cdot 2x^{2-1} = -6x$
  \item For the constant 7: The derivative of any constant is 0
\end{itemize}

Therefore:
\begin{equation}
f'(x) = 20x^3 - 6x
\end{equation}

This tells us the instantaneous rate of change of $f(x)$ at any point $x$. For example, at $x = 1$, the slope is $f'(1) = 20(1)^3 - 6(1) = 14$.

\subsection{Example 2: Product Rule}

Find the derivative of $f(x) = (2x + 1)(x^2 - 3)$.

\textbf{Solution:}

We have a product of two functions, so we use the product rule: $(uv)' = u'v + uv'$.

Let:
\begin{itemize}
  \item $u = 2x + 1$, so $u' = 2$
  \item $v = x^2 - 3$, so $v' = 2x$
\end{itemize}

Applying the product rule:
\begin{align}
f'(x) &= u'v + uv' \\
&= 2(x^2 - 3) + (2x + 1)(2x) \\
&= 2x^2 - 6 + 4x^2 + 2x \\
&= 6x^2 + 2x - 6
\end{align}

Note: We could have also expanded the original function first as $f(x) = 2x^3 + x^2 - 6x - 3$, then applied the power rule directly to get the same result.

\subsection{Example 3: Chain Rule}

Find the derivative of $f(x) = (3x^2 + 2)^5$.

\textbf{Solution:}

This is a composite function (a function inside another function), so we use the chain rule: $\dfrac{dy}{dx} = \dfrac{dy}{du} \cdot \dfrac{du}{dx}$.

Let:
\begin{itemize}
  \item Inner function: $u = 3x^2 + 2$, with $\dfrac{du}{dx} = 6x$
  \item Outer function: $y = u^5$, with $\dfrac{dy}{du} = 5u^4$
\end{itemize}

Using the chain rule:
\begin{align}
f'(x) &= \frac{dy}{du} \cdot \frac{du}{dx} \\
&= 5u^4 \cdot 6x \\
&= 5(3x^2 + 2)^4 \cdot 6x \\
&= 30x(3x^2 + 2)^4
\end{align}

At $x = 0$, we get $f'(0) = 30(0)(2)^4 = 0$, which means the tangent line is horizontal at this point.

\subsection{Example 4: Quotient Rule}

Find the derivative of $f(x) = \dfrac{x^2 + 1}{x - 1}$.

\textbf{Solution:}

We have a quotient of two functions, so we use the quotient rule: $\left(\dfrac{u}{v}\right)' = \dfrac{u'v - uv'}{v^2}$.

Let:
\begin{itemize}
  \item $u = x^2 + 1$ (numerator), so $u' = 2x$
  \item $v = x - 1$ (denominator), so $v' = 1$
\end{itemize}

Applying the quotient rule:
\begin{align}
f'(x) &= \frac{u'v - uv'}{v^2} \\
&= \frac{2x(x - 1) - (x^2 + 1)(1)}{(x - 1)^2} \\
&= \frac{2x^2 - 2x - x^2 - 1}{(x - 1)^2} \\
&= \frac{x^2 - 2x - 1}{(x - 1)^2}
\end{align}

\textbf{Note:} The domain of the derivative excludes $x = 1$ (where the denominator equals zero). Also, we can find critical points by setting $f'(x) = 0$, which occurs when $x^2 - 2x - 1 = 0$, giving $x = 1 \pm \sqrt{2}$.

\section{Conclusion}

Understanding derivatives is fundamental to calculus and has applications throughout science, engineering, and economics. By mastering the rules and concepts presented here, you'll be well-equipped to solve a wide variety of problems involving rates of change and optimization.

\end{document}
